% Created 2013-06-17 Mo 20:12
\documentclass[11pt]{article}
\usepackage{listings}
\usepackage[utf8]{inputenc}
\usepackage[T1]{fontenc}
\usepackage{fixltx2e}
\usepackage{graphicx}
\usepackage{longtable}
\usepackage{float}
\usepackage{wrapfig}
\usepackage{soul}
\usepackage{textcomp}
\usepackage{marvosym}
\usepackage{wasysym}
\usepackage{latexsym}
\usepackage{amssymb}
\usepackage{amstext}
\usepackage{hyperref}
\tolerance=1000
\author{Thorsten Jolitz}
\date{\today}
\title{\textbf{outorg-edit-buffer}}
\hypersetup{
  pdfkeywords={},
  pdfsubject={},
  pdfcreator={Emacs 24.3.1 (Org mode 8.0.3)}}
\begin{document}

\maketitle
\tableofcontents

\section{ardec}
\label{sec-1}
\subsection{meta}
\label{sec-1-1}
\subsection{man}
\label{sec-1-2}
\subsubsection{Description}
\label{sec-1-2-1}

Decomposition of a time series into latent subseries from a fitted
autoregressive model
\subsubsection{Usage}
\label{sec-1-2-2}

\begin{verbatim}
ardec(x, coef, ...)
\end{verbatim}
\subsubsection{Arguments}
\label{sec-1-2-3}

\begin{description}
\item[x] time series
\item[coef] autoregressive parameters of AR(p) modebl
\item[\ldots{}] additional arguments for specific methods
\end{description}
\subsubsection{Details}
\label{sec-1-2-4}

If an observed time series can be adequately described by an (eventually high
order) autoregressive AR(p) process, a constructive result (West, 1997) yields
a time series decomposition in terms of latent components following either
AR(1) or AR(2) processes depending on the eigenvalues of the state evolution
matrix.

Complex eigenvalues r exp(iw) correspond to pseudo-periodic oscillations as a
damped sine wave with fixed period (2pi/w) and damping factor r. Real
eigenvalues correspond to a first order autoregressive process with parameter
r.
\subsubsection{Value}
\label{sec-1-2-5}

A list with components: 

\begin{description}
\item[period] periods of latent components
\item[modulus] damping factors of latent components
\item[comps] matrix of latent components
\end{description}
\subsubsection{References}
\label{sec-1-2-6}

\begin{verse}
\cite{west1997time} \\
\cite{west1997bayesian} \\
\end{verse}
\subsubsection{Examples}
\label{sec-1-2-7}

\lstset{language=R,numbers=none}
\begin{lstlisting}
data(tempEng)
coef=ardec.lm(tempEng)$coefficients
\end{lstlisting}

warning: running the next command can be time comsuming!
\lstset{language=R,numbers=none}
\begin{lstlisting}
decomposition=ardec(tempEng,coef)
\end{lstlisting}
\subsection{R}
\label{sec-1-3}

\lstset{language=R,numbers=none}
\begin{lstlisting}
ardec <- function(x,coef, ...) {

    dat=x-mean(x)
    p=length(coef)
    ndat=length(x)

    G=matrix(nrow=p,ncol=p)

    G[1,]=coef
    G[seq(2,p),seq(1,p-1)]=diag(1,(p-1))
    G[seq(2,p),p]=0

    modulus=Mod(eigen(G)[[1]])
    lambda=2*pi/Arg(eigen(G)[[1]])

    eigenvalues=eigen(G)
    A=diag(eigenvalues[[1]])
    E=eigenvalues[[2]]
    B=solve(E)
    F=rep(NA,p)
    F[1]=1
    F[seq(2,p)]=0
    a=t(E) %*% F
    d=diag(as.vector(a))
    H=d %*% B

    Z=matrix(nrow=p,ncol=ndat)
    Z[1,]=dat
    for (i in seq(1,p-1)){
        Z[i+1,]=as.vector(filter(dat,c(rep(0,i),1), method="convolution",
        sides=1))}

    g=matrix(nrow=p,ncol=ndat)
    for (j in seq(1,p)){
        for (t in seq(1,ndat)){
            g[j,t]=H[j, ] %*% Z[,t] }}


    return(list(period=lambda,modulus=modulus,comps=g))
}
\end{lstlisting}
\subsection{data}
\label{sec-1-4}
\section{ardec.lm}
\label{sec-2}
\subsection{meta}
\label{sec-2-1}
\subsection{man}
\label{sec-2-2}
\subsubsection{Description}
\label{sec-2-2-1}

Function ardec.lm fits an autoregressive model of order p, AR(p) to a time
series through a linear least squares regression.
\subsubsection{Usage}
\label{sec-2-2-2}

\begin{verbatim}
ardec.lm(x)
\end{verbatim}
\subsubsection{Arguments}
\label{sec-2-2-3}

\begin{verbatim}
\item{x}{time series}

# \item{R}{size of sample to be simulated from posterior}

# \item{med}{logical, indicating if a median vector of autoregressive parameters
# should be computed from the simulated sample}
\end{verbatim}
\subsubsection{Value}
\label{sec-2-2-4}

\begin{verbatim}
For ardec.lm, an object of class "lm".

# For ardec.lm.bayes an Rxp matrix containing the samples of autoregressive
# coefficients as columns (if med=FALSE).

# If med=TRUE, ardec.lm.bayes returns a single column matrix containing the
# median vector of autoregressive parameters.
\end{verbatim}
\subsubsection{References}
\label{sec-2-2-5}

\cite{west1995bayesian}
\subsubsection{Examples}
\label{sec-2-2-6}

\subsection{R}
\label{sec-2-3}

\lstset{language=R,numbers=none}
\begin{lstlisting}
ardec.lm <- function(x) {

    require(stats)

    dat=x-mean(x)
    ndat=length(dat)
    ## p=order of autoregressive model from AIC (burg method)
    p=ar(dat,method="burg")[[1]]
    ## linear autoregressive model fit
    X=t(matrix(dat[rev(rep((1:p),ndat-p)+
    rep((0:(ndat-p-1)),rep(p,ndat-p)))],p,ndat-p))
    y=rev(dat[(p+1):ndat])
    fit=lm(y~-1+X, x=TRUE)

    return(fit)
}
\end{lstlisting}
\section{ardec.periodic}
\label{sec-3}
\subsection{meta}
\label{sec-3-1}
\subsection{man}
\label{sec-3-2}
\subsubsection{Description}
\label{sec-3-2-1}

Decomposition of a time series into latent subseries from a fitted
autoregressive model
\subsubsection{Usage}
\label{sec-3-2-2}

\begin{verbatim}
ardec(x, coef, ...)
\end{verbatim}
\subsubsection{Arguments}
\label{sec-3-2-3}
\subsubsection{Details}
\label{sec-3-2-4}
\subsubsection{Value}
\label{sec-3-2-5}
\subsubsection{References}
\label{sec-3-2-6}
\subsubsection{Examples}
\label{sec-3-2-7}

\subsection{R}
\label{sec-3-3}

\lstset{language=R,numbers=none}
\begin{lstlisting}
ardec.periodic <- function(x,per,tol=0.95){

    ## if(frequency(x)!=12){stop("monthly time series required")}
    ## updated 29 Apr 2013

    fit=ardec.lm(x)

    comp=ardec(x,fit$coefficients)

    if(any(comp$period > (per-tol) & comp$period < (per+tol))) {
        candidates=which(comp$period > (per-tol) & comp$period < (per+tol))
        lper=candidates[which.max(comp$modulus[candidates])]
        l=comp$period[lper]
        m=comp$modulus[lper]
        gt=Re(comp$comps[lper,]+comp$comps[lper+1,])   }

    return(list(period=l,modulus=m,component=gt))

}
\end{lstlisting}
\section{ardec.trend}
\label{sec-4}
\subsection{meta}
\label{sec-4-1}
\subsection{man}
\label{sec-4-2}
\subsubsection{Description}
\label{sec-4-2-1}

Decomposition of a time series into latent subseries from a fitted
autoregressive model
\subsubsection{Usage}
\label{sec-4-2-2}

\begin{verbatim}
ardec(x, coef, ...)
\end{verbatim}
\subsubsection{Arguments}
\label{sec-4-2-3}
\subsubsection{Details}
\label{sec-4-2-4}
\subsubsection{Value}
\label{sec-4-2-5}
\subsubsection{References}
\label{sec-4-2-6}
\subsubsection{Examples}
\label{sec-4-2-7}
\subsection{R}
\label{sec-4-3}

\lstset{language=R,numbers=none}
\begin{lstlisting}
ardec.trend <- function(x){

    options(warn=-1)

    fit=ardec.lm(x)


    comp=ardec(x,fit$coefficients)

    if(any(comp$period==Inf)){warning("no trend component")}


    if(any(comp$period ==Inf)){
        l=comp$period[which(match(comp$period,Inf)==1)[1]]
        m=comp$modulus[which(match(comp$period,Inf)==1)[1]]
        gt=Re(comp$comps[which(match(comp$period,Inf)==1 )[1],])

    }

    return(list(modulus=m,trend=gt))  

}
\end{lstlisting}
\section{tempEng}
\label{sec-5}
\subsection{meta}
\label{sec-5-1}
\subsection{man}
\label{sec-5-2}
\subsubsection{Description}
\label{sec-5-2-1}

Decomposition of a time series into latent subseries from a fitted
autoregressive model
\subsubsection{Usage}
\label{sec-5-2-2}

\begin{verbatim}
ardec(x, coef, ...)
\end{verbatim}
\subsubsection{Arguments}
\label{sec-5-2-3}
\subsubsection{Details}
\label{sec-5-2-4}
\subsubsection{Value}
\label{sec-5-2-5}
\subsubsection{References}
\label{sec-5-2-6}
\subsubsection{Examples}
\label{sec-5-2-7}
\subsection{R}
\label{sec-5-3}

\bibliographystyle{apalike}
\bibliography{ardec}
% Emacs 24.3.1 (Org mode 8.0.3)
\end{document}
